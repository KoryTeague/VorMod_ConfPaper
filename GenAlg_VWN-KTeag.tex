%% bare_conf.tex
%% V1.4b
%% 2015/08/26
%% by Michael Shell
%% See:
%% http://www.michaelshell.org/
%% for current contact information.
%%
%% This is a skeleton file demonstrating the use of IEEEtran.cls
%% (requires IEEEtran.cls version 1.8b or later) with an IEEE
%% conference paper.
%%
%% Support sites:
%% http://www.michaelshell.org/tex/ieeetran/
%% http://www.ctan.org/pkg/ieeetran
%% and
%% http://www.ieee.org/

%%*************************************************************************
%% Legal Notice:
%% This code is offered as-is without any warranty either expressed or
%% implied; without even the implied warranty of MERCHANTABILITY or
%% FITNESS FOR A PARTICULAR PURPOSE! 
%% User assumes all risk.
%% In no event shall the IEEE or any contributor to this code be liable for
%% any damages or losses, including, but not limited to, incidental,
%% consequential, or any other damages, resulting from the use or misuse
%% of any information contained here.
%%
%% All comments are the opinions of their respective authors and are not
%% necessarily endorsed by the IEEE.
%%
%% This work is distributed under the LaTeX Project Public License (LPPL)
%% ( http://www.latex-project.org/ ) version 1.3, and may be freely used,
%% distributed and modified. A copy of the LPPL, version 1.3, is included
%% in the base LaTeX documentation of all distributions of LaTeX released
%% 2003/12/01 or later.
%% Retain all contribution notices and credits.
%% ** Modified files should be clearly indicated as such, including  **
%% ** renaming them and changing author support contact information. **
%%*************************************************************************


% *** Authors should verify (and, if needed, correct) their LaTeX system  ***
% *** with the testflow diagnostic prior to trusting their LaTeX platform ***
% *** with production work. The IEEE's font choices and paper sizes can   ***
% *** trigger bugs that do not appear when using other class files.       ***                          ***
% The testflow support page is at:
% http://www.michaelshell.org/tex/testflow/



%\documentclass[conference]{IEEEtran}
\documentclass[onecolumn,draftcls]{IEEEtran}
% Some Computer Society conferences also require the compsoc mode option,
% but others use the standard conference format.
%
% If IEEEtran.cls has not been installed into the LaTeX system files,
% manually specify the path to it like:
% \documentclass[conference]{../sty/IEEEtran}
\renewcommand\IEEEkeywordsname{Index Terms}





% Some very useful LaTeX packages include:
% (uncomment the ones you want to load)


% *** MISC UTILITY PACKAGES ***
%
%\usepackage{ifpdf}
% Heiko Oberdiek's ifpdf.sty is very useful if you need conditional
% compilation based on whether the output is pdf or dvi.
% usage:
% \ifpdf
%   % pdf code
% \else
%   % dvi code
% \fi
% The latest version of ifpdf.sty can be obtained from:
% http://www.ctan.org/pkg/ifpdf
% Also, note that IEEEtran.cls V1.7 and later provides a builtin
% \ifCLASSINFOpdf conditional that works the same way.
% When switching from latex to pdflatex and vice-versa, the compiler may
% have to be run twice to clear warning/error messages.






% *** CITATION PACKAGES ***
%
\usepackage{cite}
% cite.sty was written by Donald Arseneau
% V1.6 and later of IEEEtran pre-defines the format of the cite.sty package
% \cite{} output to follow that of the IEEE. Loading the cite package will
% result in citation numbers being automatically sorted and properly
% "compressed/ranged". e.g., [1], [9], [2], [7], [5], [6] without using
% cite.sty will become [1], [2], [5]--[7], [9] using cite.sty. cite.sty's
% \cite will automatically add leading space, if needed. Use cite.sty's
% noadjust option (cite.sty V3.8 and later) if you want to turn this off
% such as if a citation ever needs to be enclosed in parenthesis.
% cite.sty is already installed on most LaTeX systems. Be sure and use
% version 5.0 (2009-03-20) and later if using hyperref.sty.
% The latest version can be obtained at:
% http://www.ctan.org/pkg/cite
% The documentation is contained in the cite.sty file itself.






% *** GRAPHICS RELATED PACKAGES ***
%
\ifCLASSINFOpdf
	\usepackage[pdftex]{graphicx}
  % declare the path(s) where your graphic files are
    \graphicspath{{Figures/}}
  % and their extensions so you won't have to specify these with
  % every instance of \includegraphics
    \DeclareGraphicsExtensions{.pdf,.jpeg,.png}
\else
  % or other class option (dvipsone, dvipdf, if not using dvips). graphicx
  % will default to the driver specified in the system graphics.cfg if no
  % driver is specified.
  % \usepackage[dvips]{graphicx}
  % declare the path(s) where your graphic files are
  % \graphicspath{{../eps/}}
  % and their extensions so you won't have to specify these with
  % every instance of \includegraphics
  % \DeclareGraphicsExtensions{.eps}
\fi
% graphicx was written by David Carlisle and Sebastian Rahtz. It is
% required if you want graphics, photos, etc. graphicx.sty is already
% installed on most LaTeX systems. The latest version and documentation
% can be obtained at: 
% http://www.ctan.org/pkg/graphicx
% Another good source of documentation is "Using Imported Graphics in
% LaTeX2e" by Keith Reckdahl which can be found at:
% http://www.ctan.org/pkg/epslatex
%
% latex, and pdflatex in dvi mode, support graphics in encapsulated
% postscript (.eps) format. pdflatex in pdf mode supports graphics
% in .pdf, .jpeg, .png and .mps (metapost) formats. Users should ensure
% that all non-photo figures use a vector format (.eps, .pdf, .mps) and
% not a bitmapped formats (.jpeg, .png). The IEEE frowns on bitmapped formats
% which can result in "jaggedy"/blurry rendering of lines and letters as
% well as large increases in file sizes.
%
% You can find documentation about the pdfTeX application at:
% http://www.tug.org/applications/pdftex





% *** MATH PACKAGES ***
%
\usepackage{amsmath}
% A popular package from the American Mathematical Society that provides
% many useful and powerful commands for dealing with mathematics.
%
% Note that the amsmath package sets \interdisplaylinepenalty to 10000
% thus preventing page breaks from occurring within multiline equations. Use:
%\interdisplaylinepenalty=2500
% after loading amsmath to restore such page breaks as IEEEtran.cls normally
% does. amsmath.sty is already installed on most LaTeX systems. The latest
% version and documentation can be obtained at:
% http://www.ctan.org/pkg/amsmath





% *** SPECIALIZED LIST PACKAGES ***
%
%\usepackage{algorithmic}
% algorithmic.sty was written by Peter Williams and Rogerio Brito.
% This package provides an algorithmic environment fo describing algorithms.
% You can use the algorithmic environment in-text or within a figure
% environment to provide for a floating algorithm. Do NOT use the algorithm
% floating environment provided by algorithm.sty (by the same authors) or
% algorithm2e.sty (by Christophe Fiorio) as the IEEE does not use dedicated
% algorithm float types and packages that provide these will not provide
% correct IEEE style captions. The latest version and documentation of
% algorithmic.sty can be obtained at:
% http://www.ctan.org/pkg/algorithms
% Also of interest may be the (relatively newer and more customizable)
% algorithmicx.sty package by Szasz Janos:
% http://www.ctan.org/pkg/algorithmicx




% *** ALIGNMENT PACKAGES ***
%
%\usepackage{array}
% Frank Mittelbach's and David Carlisle's array.sty patches and improves
% the standard LaTeX2e array and tabular environments to provide better
% appearance and additional user controls. As the default LaTeX2e table
% generation code is lacking to the point of almost being broken with
% respect to the quality of the end results, all users are strongly
% advised to use an enhanced (at the very least that provided by array.sty)
% set of table tools. array.sty is already installed on most systems. The
% latest version and documentation can be obtained at:
% http://www.ctan.org/pkg/array


% IEEEtran contains the IEEEeqnarray family of commands that can be used to
% generate multiline equations as well as matrices, tables, etc., of high
% quality.




% *** SUBFIGURE PACKAGES ***
%\ifCLASSOPTIONcompsoc
%  \usepackage[caption=false,font=normalsize,labelfont=sf,textfont=sf]{subfig}
%\else
%  \usepackage[caption=false,font=footnotesize]{subfig}
%\fi
% subfig.sty, written by Steven Douglas Cochran, is the modern replacement
% for subfigure.sty, the latter of which is no longer maintained and is
% incompatible with some LaTeX packages including fixltx2e. However,
% subfig.sty requires and automatically loads Axel Sommerfeldt's caption.sty
% which will override IEEEtran.cls' handling of captions and this will result
% in non-IEEE style figure/table captions. To prevent this problem, be sure
% and invoke subfig.sty's "caption=false" package option (available since
% subfig.sty version 1.3, 2005/06/28) as this is will preserve IEEEtran.cls
% handling of captions.
% Note that the Computer Society format requires a larger sans serif font
% than the serif footnote size font used in traditional IEEE formatting
% and thus the need to invoke different subfig.sty package options depending
% on whether compsoc mode has been enabled.
%
% The latest version and documentation of subfig.sty can be obtained at:
% http://www.ctan.org/pkg/subfig




% *** FLOAT PACKAGES ***
%
%\usepackage{fixltx2e}
% fixltx2e, the successor to the earlier fix2col.sty, was written by
% Frank Mittelbach and David Carlisle. This package corrects a few problems
% in the LaTeX2e kernel, the most notable of which is that in current
% LaTeX2e releases, the ordering of single and double column floats is not
% guaranteed to be preserved. Thus, an unpatched LaTeX2e can allow a
% single column figure to be placed prior to an earlier double column
% figure.
% Be aware that LaTeX2e kernels dated 2015 and later have fixltx2e.sty's
% corrections already built into the system in which case a warning will
% be issued if an attempt is made to load fixltx2e.sty as it is no longer
% needed.
% The latest version and documentation can be found at:
% http://www.ctan.org/pkg/fixltx2e


%\usepackage{stfloats}
% stfloats.sty was written by Sigitas Tolusis. This package gives LaTeX2e
% the ability to do double column floats at the bottom of the page as well
% as the top. (e.g., "\begin{figure*}[!b]" is not normally possible in
% LaTeX2e). It also provides a command:
%\fnbelowfloat
% to enable the placement of footnotes below bottom floats (the standard
% LaTeX2e kernel puts them above bottom floats). This is an invasive package
% which rewrites many portions of the LaTeX2e float routines. It may not work
% with other packages that modify the LaTeX2e float routines. The latest
% version and documentation can be obtained at:
% http://www.ctan.org/pkg/stfloats
% Do not use the stfloats baselinefloat ability as the IEEE does not allow
% \baselineskip to stretch. Authors submitting work to the IEEE should note
% that the IEEE rarely uses double column equations and that authors should try
% to avoid such use. Do not be tempted to use the cuted.sty or midfloat.sty
% packages (also by Sigitas Tolusis) as the IEEE does not format its papers in
% such ways.
% Do not attempt to use stfloats with fixltx2e as they are incompatible.
% Instead, use Morten Hogholm'a dblfloatfix which combines the features
% of both fixltx2e and stfloats:
%
% \usepackage{dblfloatfix}
% The latest version can be found at:
% http://www.ctan.org/pkg/dblfloatfix




% *** PDF, URL AND HYPERLINK PACKAGES ***
%
%\usepackage{url}
% url.sty was written by Donald Arseneau. It provides better support for
% handling and breaking URLs. url.sty is already installed on most LaTeX
% systems. The latest version and documentation can be obtained at:
% http://www.ctan.org/pkg/url
% Basically, \url{my_url_here}.




% *** Do not adjust lengths that control margins, column widths, etc. ***
% *** Do not use packages that alter fonts (such as pslatex).         ***
% There should be no need to do such things with IEEEtran.cls V1.6 and later.
% (Unless specifically asked to do so by the journal or conference you plan
% to submit to, of course. )
\usepackage{unicode-math}
\usepackage{color}
\usepackage{tcolorbox}
\usepackage{subcaption}


% correct bad hyphenation here
\hyphenation{net-works}


\begin{document}
%
% paper title
% Titles are generally capitalized except for words such as a, an, and, as,
% at, but, by, for, in, nor, of, on, or, the, to and up, which are usually
% not capitalized unless they are the first or last word of the title.
% Linebreaks \\ can be used within to get better formatting as desired.
% Do not put math or special symbols in the title.
%\title{Joint Base Station Selection and Adaptive Slicing in Virtualized Wireless Networks Under Uncertain User Locations}
\title{Joint Base Station Selection and Adaptive Slicing in Virtualized Wireless Networks: A Stochastic Optimization Framework}

% author names and affiliations
% use a multiple column layout for up to three different
% affiliations
\author{
\IEEEauthorblockN{Kory Teague$^1$, Mohammad J. Abdel-Rahman$^{1,2}$, and Allen B. MacKenzie$^1$\\}
\IEEEauthorblockA{$^1$Department Electrical and Computer Engineering, Virginia Tech, Blacksburg ,VA 24061, USA \\ $^2$Electrical and Energy Engineering Department, Al Hussein Technical University (HTU), Amman 11821, Jordan \\
\{koryt, mo7ammad, mackenab\}@vt.edu
}}

\iffalse
\author{\IEEEauthorblockN{Kory Teague}
\IEEEauthorblockA{School of Electrical and\\Computer Engineering\\
Virginia Tech\\
Blacksburg, VA\\
Email: koryt@vt.edu}
\and
\\
\IEEEauthorblockN{Mohammad Abdel-Rahman}
\IEEEauthorblockA{School of Electrical and\\Computer Engineering\\
Virginia Tech\\
Blacksburg, VA\\
Email: mo7ammad@vt.edu}
\and
\\
\IEEEauthorblockN{Allen MacKenzie}
\IEEEauthorblockA{School of Electrical and\\Computer Engineering\\
Virginia Tech\\
Blacksburg, VA\\
Email: mackenab@vt.edu}}
\fi

% conference papers do not typically use \thanks and this command
% is locked out in conference mode. If really needed, such as for
% the acknowledgment of grants, issue a \IEEEoverridecommandlockouts
% after \documentclass

% for over three affiliations, or if they all won't fit within the width
% of the page, use this alternative format:
% 
%\author{\IEEEauthorblockN{Michael Shell\IEEEauthorrefmark{1},
%Homer Simpson\IEEEauthorrefmark{2},
%James Kirk\IEEEauthorrefmark{3}, 
%Montgomery Scott\IEEEauthorrefmark{3} and
%Eldon Tyrell\IEEEauthorrefmark{4}}
%\IEEEauthorblockA{\IEEEauthorrefmark{1}School of Electrical and Computer Engineering\\
%Georgia Institute of Technology,
%Atlanta, Georgia 30332--0250\\ Email: see http://www.michaelshell.org/contact.html}
%\IEEEauthorblockA{\IEEEauthorrefmark{2}Twentieth Century Fox, Springfield, USA\\
%Email: homer@thesimpsons.com}
%\IEEEauthorblockA{\IEEEauthorrefmark{3}Starfleet Academy, San Francisco, California 96678-2391\\
%Telephone: (800) 555--1212, Fax: (888) 555--1212}
%\IEEEauthorblockA{\IEEEauthorrefmark{4}Tyrell Inc., 123 Replicant Street, Los Angeles, California 90210--4321}}




% use for special paper notices
%\IEEEspecialpapernotice{(Invited Paper)}




% make the title area
\maketitle

% As a general rule, do not put math, special symbols or citations
% in the abstract
\begin{abstract}
%Wireless network virtualization is a promising avenue of research for next generation 5G cellular networks.  Virtualization focuses on the concept of active resource sharing and the building of a network designed for specific demands, decreasing operational expenditures and improving demand satisfaction of the cellular network.  In this work, we investigate the problem of selecting base stations to construct a virtual network that meets the the specific demands of a service provider, and the adaptive slicing of the resources to the service provider's demand points.  We introduce a two-stage stochastic optimization problem modeling this interaction, and present two approaches for determining a solution for the model.  The first approach uses a sampling approach applied to the deterministic equivalent program of the stochastic model, providing a tight approximation with sufficient time.  The second approach uses a genetic algorithm to approximate base station selection for the network, allowing for adaptive slicing to be determined by a simpler, single-stage optimization problem.  Scenarios testing these approaches are generated using a log-normal model designed to emulate demand from real world cellular networks.  The results from these simulations present promising results that can be expounded upon with further investigation.
Wireless network virtualization is a promising avenue of research for next-generation 5G cellular networks.  Virtualization focuses on the concept of active resource sharing and the building of a network designed for specific demands, decreasing operational expenditures and improving demand satisfaction of cellular networks.  This work investigates the problem of selecting base stations (BSs) to construct a virtual network that meets the the specific demands of a service provider, and the adaptive slicing of the resources between the service provider's demand points.  A two-stage stochastic optimization framework is introduced to model the problem of joint BS selection and adaptive slicing.  We present two methods for determining a solution for the two-stage stochastic optimization model.  The first method uses a sampling approach applied to the deterministic equivalent program of the stochastic model, providing a tight approximation with sufficient computation time.  The second method uses a genetic algorithm to approximate BS selection for the network, allowing for adaptive slicing to be determined by a simpler, single-stage linear optimization problem.  To test these methods we generated a number of scenarios using a log-normal model designed to emulate demand from real world cellular networks.  The results from these simulations present promising results that can be expounded upon with further investigation.
\end{abstract}

% no keywords
\begin{IEEEkeywords}
Wireless network virtualization, resource allocation, two-stage stochastic optimization, genetic algorithm.
\end{IEEEkeywords}




% For peer review papers, you can put extra information on the cover
% page as needed:
% \ifCLASSOPTIONpeerreview
% \begin{center} \bfseries EDICS Category: 3-BBND \end{center}
% \fi
%
% For peerreview papers, this IEEEtran command inserts a page break and
% creates the second title. It will be ignored for other modes.
\IEEEpeerreviewmaketitle



\section{Introduction} \label{sec:intro}
Resource infrastructure sharing has been a common practice between mobile network operators (MNOs) to fulfill two separate needs.  First, MNOs needed to offer coverage for their users in regions where they had no infrastructure, leading to the creation of roaming agreements.  Second, the cost to maintain the wireless infrastructure became high in some areas.  Passive sharing of infrastructure, such as physical sites, tower masts, power, and air-conditioning, led to considerable cost savings \cite{1421931}.  These benefits motivated the use of more active sharing approaches, such as the reuse of backhaul and the sharing of radio access networks (RANs) in the form of base stations (BSs) and antenna systems.  The technologies developed for MNO sharing have allowed for the rise of mobile virtual network operators (MVNOs), which provide MNO-like mobile services without directly owning a physical RAN.

Wireless virtualization is one of the most promising approaches for efficient sharing of radio resources in next-generation mobile networks \cite{6824752}.  In \cite{6553675, 6571315}, the opportunities for cost saving and additional flexibility are the motivations behind introducing virtualization schemes for LTE networks, focusing on resource sharing.  Doyle \textit{et al.} \cite{6737248} presented the Network without Borders (NwoB) paradigm which broadly explores the concepts of virtualization.  The authors proposed removing the traditional constraints on spectrum and presented a virtualization-centric structure for mobile wireless networks.  In this paradigm, a service provider (SP) is an MVNO that provides a specified service to its users, which could be in the form of traditional general data, voice, and messaging services or a more specific application (e.g., emergency services, Internet of things, or video streaming).  The virtual network builder (VNB) aggregates and selects resources from resource providers (RPs) to build virtual networks designed for the needs of the SPs.  RPs, such as traditional MNOs, are the owners and maintainers of the physical resources and infrastructure that are provided for the virtualized wireless networks (VWNs) via contracts established with VNBs.

Considering the uncertainty of user equipment (demand point) locations, in this paper we consider solving two problems.  Stochastic programming provides a powerful mathematical tool to handle optimization under uncertainty.  It had been recently exploited to optimize resource allocation in various types of wireless networks operating under uncertainties (examples include~\cite{MJ_TW_13, CC_OFDMA, MJ_MECOMM_17, MJ_CCNC_16, MJ_WCNC_16, MJ_DySPAN_15, CC_video}).  We establish a stochastic optimization problem from the perspective of the VNB to determine the optimal selection of resources to be leased from the RPs and adaptively sliced and allocated to form VWNs that satisfy the demands of the SPs.  Then, we develop two approaches that would be run in the VNB to reach a near-optimal solution of the stochastic optimization problem.  We consider two optimality criteria: maximizing demand satisfaction of SP users and minimizing the cost of the BS resources to satisfy the demand.  Finally, we consider the efficacy of the approaches with a single SP with log-normal spatially-correlated demand modeled to mimic real cellular networks.

The rest of this paper is organized as follows.  In Section \ref{sec:model}, we detail and define the system model assumed for our resource allocation methods.  In Section \ref{sec:approach}, we consider our stochastic optimization problem and resource selection and demand allocation approaches.  In Section \ref{sec:sim}, we simulate the described approaches and evaluate their performance.  Finally, in Section \ref{sec:conclusion}, we discuss our conclusions and directions of future research.

% An example of a floating figure using the graphicx package.
% Note that \label must occur AFTER (or within) \caption.
% For figures, \caption should occur after the \includegraphics.
% Note that IEEEtran v1.7 and later has special internal code that
% is designed to preserve the operation of \label within \caption
% even when the captionsoff option is in effect. However, because
% of issues like this, it may be the safest practice to put all your
% \label just after \caption rather than within \caption{}.
%
% Reminder: the "draftcls" or "draftclsnofoot", not "draft", class
% option should be used if it is desired that the figures are to be
% displayed while in draft mode.
%
%\begin{figure}[!t]
%\centering
%\includegraphics[width=2.5in]{myfigure}
% where an .eps filename suffix will be assumed under latex, 
% and a .pdf suffix will be assumed for pdflatex; or what has been declared
% via \DeclareGraphicsExtensions.
%\caption{Simulation results for the network.}
%\label{fig_sim}
%\end{figure}

% Note that the IEEE typically puts floats only at the top, even when this
% results in a large percentage of a column being occupied by floats.


% An example of a double column floating figure using two subfigures.
% (The subfig.sty package must be loaded for this to work.)
% The subfigure \label commands are set within each subfloat command,
% and the \label for the overall figure must come after \caption.
% \hfil is used as a separator to get equal spacing.
% Watch out that the combined width of all the subfigures on a 
% line do not exceed the text width or a line break will occur.
%
%\begin{figure*}[!t]
%\centering
%\subfloat[Case I]{\includegraphics[width=2.5in]{box}%
%\label{fig_first_case}}
%\hfil
%\subfloat[Case II]{\includegraphics[width=2.5in]{box}%
%\label{fig_second_case}}
%\caption{Simulation results for the network.}
%\label{fig_sim}
%\end{figure*}
%
% Note that often IEEE papers with subfigures do not employ subfigure
% captions (using the optional argument to \subfloat[]), but instead will
% reference/describe all of them (a), (b), etc., within the main caption.
% Be aware that for subfig.sty to generate the (a), (b), etc., subfigure
% labels, the optional argument to \subfloat must be present. If a
% subcaption is not desired, just leave its contents blank,
% e.g., \subfloat[].


% An example of a floating table. Note that, for IEEE style tables, the
% \caption command should come BEFORE the table and, given that table
% captions serve much like titles, are usually capitalized except for words
% such as a, an, and, as, at, but, by, for, in, nor, of, on, or, the, to
% and up, which are usually not capitalized unless they are the first or
% last word of the caption. Table text will default to \footnotesize as
% the IEEE normally uses this smaller font for tables.
% The \label must come after \caption as always.
%
%\begin{table}[!t]
%% increase table row spacing, adjust to taste
%\renewcommand{\arraystretch}{1.3}
% if using array.sty, it might be a good idea to tweak the value of
% \extrarowheight as needed to properly center the text within the cells
%\caption{An Example of a Table}
%\label{table_example}
%\centering
%% Some packages, such as MDW tools, offer better commands for making tables
%% than the plain LaTeX2e tabular which is used here.
%\begin{tabular}{|c||c|}
%\hline
%One & Two\\
%\hline
%Three & Four\\
%\hline
%\end{tabular}
%\end{table}


% Note that the IEEE does not put floats in the very first column
% - or typically anywhere on the first page for that matter. Also,
% in-text middle ("here") positioning is typically not used, but it
% is allowed and encouraged for Computer Society conferences (but
% not Computer Society journals). Most IEEE journals/conferences use
% top floats exclusively. 
% Note that, LaTeX2e, unlike IEEE journals/conferences, places
% footnotes above bottom floats. This can be corrected via the
% \fnbelowfloat command of the stfloats package.


\section{System Model} \label{sec:model}
We consider a geographical area of width $X$ (m) and length $Y$ (m) that contains a set $\mathcal{S} = \left\{1,\, 2,\, \ldots,\, S\right\}$ of BSs available to be leased to the VNB by a set of $\mathcal{N} = \left\{1,\, 2,\, \ldots,\, N\right\}$ RPs.  The rate capacity of BS $s \in \mathcal{S}$ is denoted by $r_s$, its cost is denoted by $c_s$, and its coverage radius is denoted by $b_s$.

A Service Provider (SP) seeking a virtualized wireless network from the VNB is assumed to know the distribution of traffic demand within the region the VWN would cover.  It has been shown that a log-normal distribution or a mixture of log-normal distributions can approximate traffic demand in real-world cellular networks \cite{686105, 5936263}.  It has also been shown that traffic distribution is spatially correlated \cite{5936263, eigenplaces}.  We model the spatial traffic demand of a single SP using a similar, continuous form of the SSLT (Scalable, Spatially-correlated, and Log-normally distributed Traffic) model as proposed by Lee, Zhou, and Niu \cite{6554749}.

To generate this spatial distribution over the area of consideration, an initial Gaussian field, $\rho^G = \rho^G\left(x,\, y\right),\, x \in \left[0,\, X\right],\, y \in \left[0,\, Y\right]$, is generated by

\begin{equation}
\rho^G\left(x,\, y\right)=\frac{1}{L}\sum_{l=1}^L \cos\left(i_lx+\phi_l\right) \; \cos\left(j_ly+\psi_l\right)
\end{equation}

\noindent \sloppy where $\mathcal{L} \eqdef \left\{1,\, 2,\, \ldots,\, L\right\}$ is a set of the products of two cosines with angular frequencies $i_l,\, j_l\, \textasciitilde\, \mathcal{U}\left(0,\, \omega_{\max}\right),\, l \in \mathcal{L}$ and phases $\phi_l,\, \psi_l\, \textasciitilde\, \mathcal{U}\left(0,\, 2\pi\right),\, l \in \mathcal{L}$.  As $L$ increases, $\rho^G$ approaches a Gaussian random field with a spatial autocorrelation dependent on $\omega_{\max}$ according to the central limit theorem.

The approximate Gaussian distribution $\rho^G$ is then normalized to $\rho^S = \rho^S\left(x,\, y\right), x \in [0,\, X], y \in [0,\, Y]$, which has a standard normal distribution.  The final log-normal distribution, $\rho = \rho\left(x,\, y\right), x \in [0,\, X], y \in [0,\, Y]$, is determined by assigning location and scale parameters $\mu$ and $\sigma$

\begin{equation}
\rho\left(x,\, y\right) = \exp\left(\sigma \; \rho^S\left(x,\, y\right)+\mu\right)
\end{equation}

$\rho\left(x,\, y\right)$ can be sampled over the space into individual pixels as per Lee with each pixel's value indicating the number of homogeneous demand points within the pixel \cite{6554749}.  In contrast, we allow $\rho\left(x,\, y\right)$ to provide a continuous, spatially-correlated log-normal distribution depicting the demand density over the region for the SP.

Let $\mathcal{M} \eqdef \{1,\, 2,\, \ldots,\, M\}$ be the set of the SP's demand points seeking to connect to the VWN; the value of total traffic demand at each point is denoted by $d_m$.  Further, let $u_{ms} \in [0,\, 1],\, m \in \mathcal{M},\, s \in \mathcal{S}$, represent the normalized capacity (with respect to $r_s$) of BS $s$ at point $m$, i.e., the normalized maximum rate that a user can receive at point $m$ from BS $s$.  $u_{ms} = 0$ when $m$ is outside the coverage area of $s$ and $u_{ms} = 1$ when $m$ is within a small distance of $s$.  The specific position of point $m$, and therefore the value of $u_{ms}$, is determined stochastically by the demand field $\rho$.

We assume that a BS $s \in \mathcal{S}$ can be allocated between multiple demand points, and $\delta_{ms} \in [0,\, r_s],\, m \in \mathcal{M},\, s \in \mathcal{S}$, represents the rate of BS $s$ that is allocated to point $m$.

Throughout this paper, stochastic variables will be differentiated from deterministic variables with a tilde (\textasciitilde) placed above the symbol.

\section{Solution Approach} \label{sec:approach}
In this section, we detail our approaches for selecting the subset of resources within $\mathcal{S}$ to create a virtual network with the minimum cost while allocating the selected resources to the demand points within the region such that it maximizes demand satisfaction.  First, we formulate the problem of BS selection and their allocation to individual demand points as a two-stage stochastic optimization problem.  Second, the optimization problem is converted to its deterministic equivalent program to remove the stochastic variables then approximated using a sampling method.  Finally, a genetic algorithm is used as a separate approximation to the BS selection portion of the problem.  In Section \ref{sec:sim}, we will consider the efficacy of these latter two approaches.

\subsection{Problem Formulation} \label{subsec:stoch}
We formulate the presented problem as a two-stage stochastic optimization problem.  We introduce $z_s, s \in \mathcal{S}$ as a binary decision variable defined as

\[ z_s =
	\begin{cases}
		1,& \text{if BS $s$ is selected for the created network,}\\
		0,& \text{otherwise}
	\end{cases}
\]

To balance the interest of maximizing demand satisfaction against minimizing cost, we introduce the positive real number $\alpha$ as a weighting coefficient between the two stages.

\vspace{5mm}
\begin{tcolorbox}[title = Problem 1 (Two-Stage Stochastic Optimization Problem)]
\begin{align}
& \underset{\left\{z_s,\, s \in \mathcal{S}\right\}}{\text{minimize}} \left\{ \sum_{s \in \mathcal{S}} c_s \; z_s + \alpha \mathbb{E}\left[ h\left( \textbf{\textit{z}},\, \mbox{$\tilde{\textbf{\textit{u}}}$} \right) \right] \right\} \label{eq:P1S1}\\
& \text{subject to:}  \nonumber \\
& \hspace{0.4in} z_s \in \{0, 1\}, \forall s \in \mathcal{S} \label{eq:P1S1C1}
\end{align}
where $h(\mbox{\textbf{\textit{z}}}, \mbox{$\tilde{\textbf{\textit{u}}}$})$ is the optimal value of the second-stage problem, which is given by:
\begin{align}
& \underset{\left\{\substack{\delta_{ms}, m \in \mathcal{M}, s \in \mathcal{S}}\right\}}{\mathrm{minimize}} \left\{ - \sum_{m \in \mathcal{M}} \sum_{s \in \mathcal{S}} \delta_{ms} \; \tilde{u}_{ms} \right\} \label{eq:P1S2}\\
& \text{subject to:}  \nonumber \\
& \hspace{0.4in} z_s = \mathbb{1}_{\left\{\sum_{m \in \mathcal{M}} \delta_{ms} > 0\right\}}, \forall s \in \mathcal{S} \label{eq:P1S2C1}\\
& \hspace{0.4in} \sum_{s \in \mathcal{S}} \delta_{ms} \; \tilde{u}_{ms} \leq d_m, \forall m \in \mathcal{M} \label{eq:P1S2C2}\\
& \hspace{0.4in} \sum_{m \in \mathcal{M}} \delta_{ms} \leq r_s, \forall s \in \mathcal{S} \label{eq:P1S2C3}
\end{align}
\end{tcolorbox}

The first stage objective function (\ref{eq:P1S1}) minimizes the total cost of the selected network in context to that network's ability to satisfy the demand contained within the region, as determined by $\rho$.  The second stage objective function (\ref{eq:P1S2}) maximizes demand satisfaction by maximizing the total demand allocated to the resources comprising the network, as specified by $\delta_{ms}$ as the decision variable of the second stage.  In this context, we define demand satisfaction as the ratio of the total demand allocated to the selected network to the total demand contained within the region.

Constraints (\ref{eq:P1S1C1}), (\ref{eq:P1S2C1}), and (\ref{eq:P1S2C3}) implement the defined ranges and values of the decision variables $z_s$ and $\delta_{ms}$, with (\ref{eq:P1S2C1}) ensuring that demand is allocated only to selected resources.  For constraint (\ref{eq:P1S2C1}), $\mathbb{1}_{\{*\}}$ is defined by

\[ \mathbb{1}_{\{*\}} =
	\begin{cases}
		1,& \text{if condition $*$ is true,}\\
		0,& \text{otherwise}
	\end{cases}
\]

Constraint (\ref{eq:P1S2C2}) ensures a demand point $m \in \mathcal{M}$ is not allocated more resource capacity than it demands.

\subsection{Deterministic Equivalent Reformulation} \label{subsec:dep}

In order to solve the two-stage stochastic optimization formulation (Problem 1), we need to convert it to a deterministic equivalent program (DEP) that does not contain any stochastic variables (only deterministic variables) \cite{stochprogramming}.

Let $ \Omega \eqdef \{1,\, 2,\, \ldots,\, |\Omega|\} $ be defined as the set of discrete scenarios, each of which contains a sampled version of the stochastic variables within Problem 1.  As $|\Omega|$ approaches infinity, $\Omega$ contains the entire scope of the stochastic variables.  The probability a given scenario $\omega \in \Omega$ occurs is denoted by $p^{(\omega)},\, \omega \in \Omega$, where $\sum_{\omega \in \Omega} p^{(\omega)} = 1$.  Variables that are dependent on $\Omega$ are shown with a superscript ($\omega$) with the specific scenario it is dependent on indicated by $\omega$.

\vspace{5mm}
\begin{tcolorbox}[title = Problem 2 (Deterministic Equivalent Program of Problem 1)]
\begin{align}
& 
\underset{\left\{ \substack{
	z_s,\, \delta_{ms}^{(\omega)},\\
	s \in \mathcal{S},\, m \in \mathcal{M},\\
	\omega \in \Omega} \right\}} {\text{minimize}}
\sum_{s \in \mathcal{S}} c_s \; z_s - \alpha \sum_{\omega \in \Omega} p^{(\omega)} \left( \sum_{m \in \mathcal{M}} \sum_{s \in \mathcal{S}} \delta_{ms}^{\{\omega\}} \; u_{ms}^{\{\omega\}} \right) \label{eq:P2}\\
& \text{subject to:}  \nonumber \\
& \sum_{s \in \mathcal{S}} \delta_{ms}^{\{\omega\}} \; u_{ms}^{\{\omega\}} \leq d_m,\, \forall m \in \mathcal{M},\, \forall \omega \in \Omega \label{eq:P2C1}\\
& \sum_{m \in \mathcal{M}} \delta_{ms}^{\{\omega\}} \leq r_s \; z_s,\, \forall s \in \mathcal{S},\, \forall \omega \in \Omega \label{eq:P2C2}\\
& z_s \in \{0,\, 1\},\, \forall s \in \mathcal{S} \label{eq:P2C3}
\end{align}
\end{tcolorbox}

The objective function (\ref{eq:P2}) combines both objective functions (\ref{eq:P1S1}) and (\ref{eq:P1S2}) of the initial formulation into a deterministic form.  Constraints (\ref{eq:P2C1}) and (\ref{eq:P2C2})	ensure demand is not overallocated and is only allocated to selected resources and within capacity for all scenarios.

While Problem 2 provides an equivalent deterministic form of Problem 1, it only remains so while $\Omega$ contains the entire scope of the stochastic variables (i.e., $\Omega$ contains every scenario that the stochastic variables can take with their associated probabilities occurring).  For Problem 2 to be in a generally tractable form, $\Omega$ needs to be sampled.  We sample $\Omega$ with a finite number of scenarios, $|\Omega|$.  With sufficiently large $|\Omega|$, $\Omega$ approaches a tight approximation of the scope of the stochastic variables.  Within each scenario $\omega \in \Omega$, the SSLT demand field $\rho$ is sampled to provide a set of \textit{M} discrete demand points.  Each sampling of $\rho$ is generated by creating a non-stationary 2D Poisson point process (PPP) with \textit{M} points over the region using $\rho$ as the spatial intensity function.  To generate this non-stationary PPP, we use an acceptance-rejection method.  That is, each point of a stationary PPP with an intensity of $\rho_{\max} = \max_i\rho\left(x_i,\, y_i\right)$ is retained with probability $\frac{\rho\,\left(x_i,\, y_i\right)}{\rho_{\max}}$, where $x_i$ and $y_i$ are the x- and y-coordinates of the $i^{th}$ point of the stationary PPP.

\subsection{Adaptive Slicing within a Formed VWN} \label{subsec:slice}

After the solution to the sampled DEP of Section \ref{subsec:dep} has been found, the VNB has determined the joint BS selection that forms the VWN and a proposed resource slicing of considered possible scenarios, $\Omega$, that allocates the resources to the SP's demand points.  Since $|\Omega|$ is not infinite, any given scenario present in the formed VWN is exceedingly unlikely to be an element of $\Omega$.  Further, as demand points move between BSs or enter or exit the VWN, a new scenario $\omega \notin \Omega$ is formed with a possibly new set $\mathcal{M}' \eqdef \{1,\, 2,\, \ldots,\, M'\}$.  The VWN must adapt its resource slicing to these new demand points to maintain maximal demand satisfaction.  With the VWN built, the joint BS selection, $z_s$, becomes a constant of the network, simplifying Problem 2 to a single-stage optimization problem

\vspace{5mm}
\begin{tcolorbox}[title = Problem 3 (Deterministic Adaptive Slicing)]
\begin{align}
& 
\underset{\left\{ \substack{
	\delta_{ms}^{\{\omega\}},\,	s \in \mathcal{S},\\
	m \in \mathcal{M}',\, \omega \in \Omega'} \right\}} {\text{minimize}}
- \sum_{\omega \in \Omega'} p^{(\omega)} \left( \sum_{m \in \mathcal{M}'} \sum_{s \in \mathcal{S}} \delta_{ms}^{\{\omega\}} \; u_{ms}^{\{\omega\}} \right) \label{eq:P3}\\
& \text{subject to:}  \nonumber \\
& \sum_{s \in \mathcal{S}} \delta_{ms}^{\{\omega\}} \; u_{ms}^{\{\omega\}} \leq d_m,\, \forall m \in \mathcal{M}',\, \forall \omega \in \Omega' \label{eq:P3C1}\\
& \sum_{m \in \mathcal{M}'} \delta_{ms}^{\{\omega\}} \leq r_s \; z_s,\, \forall s \in \mathcal{S},\, \forall \omega \in \Omega' \label{eq:P3C2}
\end{align}
\end{tcolorbox}

\noindent where $\Omega' \eqdef \{1,\, 2,\, \ldots,\, |\Omega|'\}$ is the set of $|\Omega|'$ scenarios independent of the original set $\Omega$.  In practice, $|\Omega|' = 1$ with $p^{(1)} = 1$, as only the currently existing scenario of the network is of interest for slicing the resources at that moment.  Higher values of $|\Omega|'$ are useful for simulating multiple scenarios with a homogeneous $\mathcal{M}'$.  It is worth noting that Problem 3 is more tractable than Problem 2 as it only contains the single decision variable for resource slicing, simplifying the objective function (\ref{eq:P3}) and constraint (\ref{eq:P3C2}).

%\textit{\textcolor{blue}{Note: It just occurred to me that this problem can be further simplified such that the set $\mathcal{S}$ is trimmed to remove all BSs $s \in \mathcal{S}$ where $z_s = 0$, creating a new set $\mathcal{S}'$ with no $z_s$ present (as $z_s = 1, \forall s \in \mathcal{S}'$). This would decrease the size of the problem in the situation it begins to become intractable.  It is implemented as displayed in Problem 3, but this would be an universal improvement.  Might be worth implementing for thesis.}}

\subsection{Genetic Algorithm Approach} \label{subsec:ga}
The full sampled DEP formulation is notably intractable as its components increase in size.  Most importantly, the accuracy of the sampled DEP is directly dependent on the size of $\Omega$, directly causing a trade off between the accuracy of the sampled DEP and its computability in a reasonable amount of time.  In this subsection, we reformulate the problem of joint BS selection for the VWN as a genetic algorithm, circumventing the need to discretize demand or to establish $\Omega$, thereby simplifying the original problem into a more scalable form.

A genetic algorithm is an iterative metaheuristic in which an approximate solution to a given optimization problem is arrived at via a series of progressive generations.  Each generation contains a number of candidate solutions, called individuals, each of which is defined by a chromosome.  During a given generation, each individual is assessed a fitness heuristic based on its chromosome.  Then individuals are selected at random, with more fit individuals being selected with greater probability.  Pairs of selected individuals will crossover with probability $p_\text{xov}$, a process similar to genetic recombination in biology.  The resulting chromosomes then have probability $p_\text{mut}$ to mutate, altering the chromosome slightly.  Once enough new individual chromosomes have been selected and possibly undergone crossover and mutation, this set of new individuals, called children, forms the next generation to repeat the process.  %(\textcolor{blue}{\textit{Trim paragraph and cite someone else?}})

For the genetic algorithm, $\rho$ is not sampled for discrete demand points.  Instead, we assume that all demand over the region is allocated to the closest resource.  The subset of $\mathcal{S}$, $\mathcal{S}'$, that is selected for a given possible VWN forms a Voronoi tessellation from the point locations of the selected resources.  The total demand allocated to a selected resource $s \in \mathcal{S}' \subseteq \mathcal{S}$ is $\iint_{V_s} \rho\left(x,\, y\right) \,dx \,dy$, where $V_s$ is the region bounded by the cell of resource \textit{s} in the Voronoi tessellation.  If the total demand allocated to \textit{s} exceeds $r_s$, \textit{s} is considered to be \textit{overcapacity}.  If $V_s$ is not wholly contained within the coverage area of resource \textit{s}, \textit{s} is considered to be \textit{overcoverage}.

Let $\mathcal{G} \eqdef \left\{1,\, 2,\, \ldots,\, G\right\}$ be the set of generations used in the genetic algorithm and $\mathcal{I}_g \eqdef \left\{1,\, 2,\, \ldots,\, I\right\}, g \in \mathcal{G}$ be the set of individuals within generation \textit{g}.  Each individual $i \in \mathcal{I}_{g \in \mathcal{G}}$ has a binary chromosome $z^{\{ig\}}$ of length \textit{S}.  $z_s^{\{ig\}}, s \in \mathcal{S}$, denoting each individual bit of the chromosome, is defined as follows:

\[ z_s^{\{ig\}} =
	\begin{cases}
		1,& \text{if BS $s$ is selected for the VWN for individual $i$ in generation $g$,}\\
		0,& \text{otherwise}
	\end{cases}
\]

The fitness heuristic of each individual chromosome, $z^{\{ig\}}$, is assessed as the reciprocal of the chromosome's cost, which is defined as

\begin{equation} \label{eq:GAFit}
\text{fitness}\left(z^{\{ig\}}\right) = \frac{1}{\text{cost}\left(z^{\{ig\}}\right)}
\end{equation}
\begin{equation} \label{eq:GACost}
\text{cost}\left(z^{\{ig\}}\right) = \sum_{s \in \mathcal{S}} \left( c_s \; z_s^{\{ig\}} + c_\text{cov} \; \mathbb{1}_{\left\{ V_s \nsubseteq R_s \right\}} + \left(c_\text{cap}^g - 1\right) \; \max\left( 0,\, \iint_{R_s} \rho\left(x,\, y\right)\, dx\, dy - r_s \right) \right)
\end{equation}

\noindent where $R_s$ is the coverage area region of resource $s \in \mathcal{S}$.

The cost function (\ref{eq:GACost}) indicates cost increases not only based on the cost of the resources selected, but also with imperfection costs $c_\text{cov}$ and $c_\text{cap}$, the costs of a selected resource being overcoverage or overcapacity, respectively.  The overcapacity cost grows with each successive generation.  For early generations, this allows for imperfect solutions to temporarily exist to seed later generations and improve diversity to increase the probability of finding a better final approximate solution.

Elitism is used, where the \textit{n} most fit individuals of a given generation are automatically selected without crossover or mutation to be the first children of the next generation.  Selection occurs via the roulette wheel selection method.  Every individual \textit{i} of a given generation \textit{g} has a probability of being selected given by

% Might need citation or reasoning for roulette-wheel selection method

\[
\frac{\text{fitness}\left( z^{\{ig\}} \right)}{\sum_{i \in \mathcal{I}} \text{fitness}\left( z^{\{ig\}} \right)}
\]

When crossover is performed on selected individuals, it is via the uniform crossover method with a mixing ratio of 0.5.  That is, if two selected parent individuals crossover, each equivalent bit in the parents will swap with a probability of 50\%.  It has been suggested that uniform crossover is more exploratory than \textit{n}-point crossover \textcolor{blue}{\textit{(cite)}}.  Mutation occurs on a bit-by-bit level, with each bit mutating (i.e., flipping) with probability $\frac{1}{S}$.  The uniqueness property is then enforced on the resulting children to ensure diversity; if a child chromosome is identical to another child chromosome in the next generation, the child is discarded and a new child generated, ensuring that each individual of any given generation is unique within that generation.

The genetic algorithm iterates for a number of generations \textit{G}.  If the genetic algorithm settles on a single individual for a number of continuous generations, $G_\text{halt}$, it will halt and present that individual's chromosome as the final approximate solution for $z_s$.  Otherwise, the chromosome of the fittest individual of generation \textit{G} determines $z_s$.

The genetic algorithm only determines an approximate solution to the BS selection forming the VWN, informing the VNB of which BSs to obtain from the RPs.  With this selection, $z_s$, the SP's demand points can be dynamically allocated resource slices as described by Problem 3 in Section \ref{subsec:slice}.

\section{Simulation and Evaluation} \label{sec:sim}
In this section, we evaluate the sampled DEP and genetic algorithm approaches as approximations of Problem 1.  We will compare the cost, demand satisfaction, and time to generate of the resultant networks.

\subsection{Setup} \label{subsec:setup}

\begin{table} \centering
\caption{Numerical Values of Relevant Parameters}
\begin{tabular}{|c|c|} 
\hline
\textbf{Parameter} & \textbf{Value} \\
\hline
\hline 
Width of Geographic Area ($X$) & 2 km \\
\hline
Height of Geographic Area ($Y$) & 2 km \\
\hline
Number of BSs ($S$) & 60 \\ 
\hline 
Number of Demand Points ($M$) & 75 \\ 
\hline 
Number of Scenarios ($|\Omega|$) & 25 \\ 
\hline 
BS cost ($c_s, \forall s \in \mathcal{S}$) & 1 \\ 
\hline 
BS capacity ($r_s, \forall s \in \mathcal{S})$ & 1.50 Mbps \\ 
\hline
BS range ($b_s, \forall s \in \mathcal{S}$) & 500 m \\
\hline 
Demand Point Demand ($d_m, \forall m \in \mathcal{M}$) & 0.178 Mbps \\ 
\hline 
Set of Two-Stage Model Weights ($\alpha$) & $\{5,\, 10,\, \ldots,\, 100\}$ \\ 
\hline 
\hline
SSLT Approximation Depth ($L$) & 50 \\ 
\hline
SSLT Maximum Angular Frequency ($\omega_{\max}$) & $\frac{2 \; \pi}{30}$ \\
\hline 
SSLT Location Parameter ($\sigma$) & 0 \\ 
\hline 
SSLT Scale Parameter ($\mu$) & 1 \\ 
\hline
Pixel Grid Size & 100 x 100, 20 m side\\
\hline 
\hline
Maximum Number of Generations ($G$) & 3000 \\ 
\hline
Minimum Number of Generations & 300 \\
\hline
Number of Unchanged Generations Before Halt ($G_\text{halt}$) & 150 \\
\hline 
Number of Individuals per Generation ($I$) & 80 \\ 
\hline
Number of Elite Individuals per Generation & 4 \\
\hline 
Probability of Crossover ($p_\text{xov}$) & 0.7 \\ 
\hline
Probability of Mutation per bit ($p_\text{mut}$) & $\frac{1}{S} = 0.0167$ \\
\hline 
Overcoverage Cost ($c_\text{cov}$) & 3 \\
\hline
Overcapacity Cost ($c_\text{cap}$) & 1.015 \\
\hline
\end{tabular}
\label{tab:simval}
\end{table}

\begin{figure}[h]
\begin{subfigure}{.5\textwidth}
	\centering
	\includegraphics[width=.8\linewidth]{Figures/BSLocationsVoronoi}
	\caption{RP BS Locations in $\mathcal{S}$ with associated Voronoi tesselation}
	\label{fig:BSLocVor}
\end{subfigure}
\begin{subfigure}{.5\textwidth}
	\centering
	\includegraphics[width=.8\linewidth]{Figures/SSLTnsPPP_demandpointreal}
	\caption{SP SSLT demand density field with one scenario of SP demand points}
	\label{fig:SSLTDPReal}
\end{subfigure}
\caption{Visualization of network area.  (\ref{fig:BSLocVor}) visualizes the resources available to VNB for creating the VWN, and (\ref{fig:SSLTDPReal}) visualizes the demand to be satisfied by the VWN.}
\label{fig:NetworkArea}
\end{figure}

Unless stated otherwise, we use the default parameter values shown in Table \ref{tab:simval}.  BS locations are determined as a stationary PPP.  Demand point locations are generated independently for each scenario as a non-stationary PPP using $\rho\left(x,\, y\right),\, x \in [0,\, X],\, y \in [0,\, Y]$, as the spatial intensity function, as described in Section \ref{sec:model}.  Fig. \ref{fig:NetworkArea} provides a visualization of the simulation network area.  (\ref{fig:BSLocVor}) shows the BS locations of $\mathcal{S}$ with the associated Voronoi Tessellation showing the coverage areas of the BSs when all are active with respect to the genetic algorithm.  (\ref{fig:SSLTDPReal}) shows the SSLT demand density field with one example scenario of demand points, which acts as a single sample of the demand density field.  To compute $u_{ms}^{\{\omega\}}$, it is assumed that there is perfect propagation between the demand points and BSs.  Unlike as described in Section \ref{sec:model}, $u_{ms}^{\{\omega\}} = 1$ if the distance between demand point $m \in \mathcal{M}$ of scenario $\omega \in \Omega$ and BS $s \in \mathcal{S}$ is less than $b_s$, and 0 otherwise.  To compute the integral of the fitness function (\ref{eq:GACost}), $\rho\left(x,\, y\right)$ is discretized into a grid of congruent pixels, and the demands of all pixels within a Voronoi cell of interest are summed together.

We ran our simulations on an Intel Core i7-4790K 4.00 GHz 4 real/8 virtual core CPU with 16 GB of DDR3 RAM.  We used CPLEX \cite{Cplex} to solve the sampled DEP optimization problems and we used MATLAB to simulate the genetic algorithm and to generate the demand field and stochastic data (i.e., $\rho\left(x,\, y\right)$ and $u_{ms}^{\{\omega\}}$).  During the simulations, extraneous processes were culled to allow maximal use of computer resources.  Average values for the performance of the genetic algorithm are provided from 50 independent runs using the identical data set except for the set of initial individuals.  The sampled DEP solutions were solved across multiple values of $\alpha$ as cost, time, and demand satisfaction are directly dependent on $\alpha$.

\subsection{Results} \label{subsec:results}

\begin{figure}[h]
\begin{subfigure}{.5\textwidth}
	\centering
	\includegraphics[width=.8\linewidth]{Figures/ComparisonTime}
	\caption{True CPU Time}
	\label{fig:CPUTimeComp}
\end{subfigure}
\begin{subfigure}{.5\textwidth}
	\centering
	\includegraphics[width=.8\linewidth]{Figures/ComparisonTimeAdjusted}
	\caption{Adjusted (Pseudo-Real) Run Time}
	\label{fig:CPUTimeCompAdj}
\end{subfigure}
\caption{Run time comparison.  The DEP run time is the solid blue line, while the minimum, average, and maximum genetic algorithm run times are provided as dot-dashed, solid, and dashed red reference lines, respectively.}
\label{fig:CPUTime}
\end{figure}

In Fig. \ref{fig:CPUTime} is a comparison of the time to run.  (\ref{fig:CPUTimeComp}) shows the absolute overall CPU time taken to converge to a solution in both the DEP and genetic algorithm.  (\ref{fig:CPUTimeCompAdj}) shows an adjusted form of the run time the DEP took to terminate to a single solution.  CPLEX is capable of parallelizing across the 8 CPU cores, allowing for the real run time to be, at minimum, one-eighth the CPU run time.  With this adjustment providing an improvement to the sampled DEP run time, the genetic algorithm converges in approximately the same time as the sampled DEP for higher levels of $\alpha$, when a single solution dominates.  Without the adjustment, the genetic algorithm outperforms the sampled DEP, converging to a solution in approximately 13\% of the time for $\alpha \geq 30$ and in approximately 2\% of the time for $\alpha \in \{20,\, 25\}$.

\begin{figure}
\begin{subfigure}{.5\textwidth}
	\centering
	\includegraphics[width=.8\linewidth]{Figures/ComparisonCost}
	\caption{VWN costs}
	\label{fig:VWNCompCost}
\end{subfigure}
\begin{subfigure}{.5\textwidth}
	\centering
	\includegraphics[width=.8\linewidth]{Figures/ComparisonSatisfaction}
	\caption{SP demand point satisfaction}
	\label{fig:VWNCompSatis}
\end{subfigure}
\caption{Comparison of VWN costs and SP demand point satisfaction.  Note that cost and demand satisfaction show a direct correlation.}
\label{fig:VWNComp}
\end{figure}

The trade off for the genetic algorithm's improved run time is that the solution provided is less optimal than the sampled DEP, as indicated by an increased cost for the VNB to build the VWN.  Fig. \ref{fig:VWNCompCost} compares the increasing cost of the sampled DEP as $\alpha$ increases with the cost of the various genetic algorithm solutions.  On average, the genetic algorithm incurs a 36\% increased cost in selecting the BSs for the VWN.  At minimum, the incurred cost is only 20\% than the sampled DEP, which implies the genetic algorithm might be terminating early, and a tighter solution might be found by increasing $G_\text{halt}$.  It should also be noted that one unit of cost is one additional BS being selected for the VWN, and the sampled DEP selections for $\alpha \geq 30$ have a cost of only 10 BSs.  Any variance that incurs one additional BS for the genetic algorithm incurs 10\% increased cost.  Increasing the number of BSs required to comprise the VWN would introduce additional granularity in $\mathcal{S}$ that might decrease the inefficiency of the genetic algorithm.  This was not done as this data set was chosen specifically so the sampled DEP would terminate within 15 minutes (i.e., in a reasonable amount of time); increasing the number of BSs available to the VNB drastically increases the time it would take the sampled DEP to converge to a solution.

There is a direct correlation between the number of BS in the VWN and its capability for satisfying demand.  As the cost increases of the sampled DEP solution, more BS are selected, and the demand satisfaction the sampled DEP solution similarly trends towards 100\%, as shown in Fig. \ref{fig:VWNComp}.  Because of this, the genetic algorithm solutions have a very high demand satisfaction, averaging 99.9\% demand satisfaction, even reaching 100\% for some solutions.  The most expensive 10-BS sampled DEP solutions reach 99.2\% demand satisfaction when slicing the same set of resources that determined the sampled DEP BS selection.

\begin{figure}
	\centering
	\includegraphics[width=0.8\linewidth]{Figures/ComparisonSatisfactionEval}
	\caption{Comparison of demand satisfaction with 50 scenarios of 200 SP demand points}
	\label{fig:VWNEvalCompSatis}
\end{figure}

When the set of demand points change to a scenario no longer in $\Omega$, the sampled DEP performs very similarly.  Fig. \ref{fig:VWNEvalCompSatis} shows the demand satisfaction for both the sampled DEP and genetic algorithm BS selections when sliced to a new set of scenarios.  Here, the number of demand points increases to 200 points per scenario, each with 66.8 kbps rate demand, across 50 independent scenarios.  The demand satisfaction trend of the sampled DEP BS selection follows very closely to the original set of scenarios but hits a maximum of 99.0\% demand satisfaction with 10 BSs.  In comparison, the SP demand point scenarios are far more beneficial to the genetic algorithm, which reaches greater than 99.99\% demand satisfaction for all generated VWNs.  This is expected as a side effect of the increased number of points and scenarios more accurately describing a sampling of the original SSLT demand density field of the SP, $\rho\left(x,\, y\right)$.

\section{Conclusion} \label{sec:conclusion}

In this work, we studied the problem of determining the optimal joint BS selection for a virtual network and the adaptive slicing of the network to the SP demand points.  We introduced a two-stage stochastic optimization problem which minimizes the cost of base station selection for a virtual network while maximizing demand satisfaction of the network's demand points.  We then investigated two approaches that provide a solvable form for the stochastic model: a deterministic equivalent program of the stochastic problem which is simplified to a tractable form with a sampling method, and a genetic algorithm that determines base station selection so that slicing can be found via a single-stage optimization problem.

The sampled DEP is considered to be closer to the original problem, and will provide a tight approximate solution to the original DEP or stochastic model with a sufficiently large set of scenarios and sufficient time, but trends to the intractable as these are reached.  The genetic algorithm is a hybrid approach intending to simplify the first stage of the model by using a metaheuristic.  All demand, known from a continuous demand density model, within a resource's cell in a Voronoi tessellation is considered allocated to that resource, a simpler process than attempting to allocate individual demand points to the entire pool of resources.  The advantage of this approach is that it simplifies the search space of the original two-stage model to a single-stage model which only handles slicing demand points to the selected resources.  This provides a faster solution with the introduction of error.

The simulation results indicate that the genetic algorithm approach may be an adequate avenue for a solution.  This approach trends to a solution in 2\% of the time, incurring 20\% increased error.  While the error is significant, it could decrease with larger, more computationally-intensive data sets that would provide more resolution to both the resource and demand sets, allowing for less error to be incurred for small variations.  It is important to note that increasing the size of the data set incurs a larger time cost on the sampled DEP approach than on the genetic algorithm approach.  Analyzing these larger data sets, improving on the genetic algorithm formulation, or utilizing other metaheuristic algorithms could improve on these results and would be obvious avenues for further investigation.



% conference papers do not normally have an appendix


% use section* for acknowledgment
%\section*{Acknowledgment}


%The authors would like to thank...





% trigger a \newpage just before the given reference
% number - used to balance the columns on the last page
% adjust value as needed - may need to be readjusted if
% the document is modified later
%\IEEEtriggeratref{8}
% The "triggered" command can be changed if desired:
%\IEEEtriggercmd{\enlargethispage{-5in}}

% references section

% can use a bibliography generated by BibTeX as a .bbl file
% BibTeX documentation can be easily obtained at:
% http://mirror.ctan.org/biblio/bibtex/contrib/doc/
% The IEEEtran BibTeX style support page is at:
% http://www.michaelshell.org/tex/ieeetran/bibtex/
%\bibliographystyle{IEEEtran}
% argument is your BibTeX string definitions and bibliography database(s)
%\bibliography{IEEEabrv,../bib/paper}
%
% <OR> manually copy in the resultant .bbl file
% set second argument of \begin to the number of references
% (used to reserve space for the reference number labels box)

\bibliography{GenAlg_VWN-KTeag}
\bibliographystyle{IEEEtran}


% that's all folks
\end{document}


